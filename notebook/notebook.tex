\documentclass[oneside]{labbook}

\usepackage[bottom=10em]{geometry} % Reduces the whitespace at the bottom of the page so more text can fit
\usepackage{amsmath} % For typesetting math
\usepackage{graphicx} % Required for including images
\usepackage{lipsum}
\usepackage[english]{babel} % English language
\usepackage[utf8]{inputenc} % Uses the utf8 input encoding
\usepackage[T1]{fontenc} % Use 8-bit encoding that has 256 glyphs
\usepackage[osf]{mathpazo} % Palatino as the main font
% \linespread{1.1}\selectfont % Palatino needs some extra spacing, here 5% extra
\usepackage[nouppercase,headsepline]{scrpage2} % Provides headers and footers configuration
\usepackage{booktabs,array} % Packages for tables
\usepackage{enumitem}

\newcommand{\obj}[1]{\textbf{Objective:} #1}
\DeclareFixedFont{\textcap}{T1}{phv}{bx}{n}{1.5cm} % Font for main title: Helvetica 1.5 cm
\DeclareFixedFont{\textaut}{T1}{phv}{bx}{n}{0.8cm} % Font for author name: Helvetica 0.8 cm
%------------------------------------------------------------------------------
%	DEFINITION OF EXPERIMENTS
%------------------------------------------------------------------------------
\newexperiment{gillespie}{Gillespie algorithm proof-of-concept}
\newexperiment{conjug}{Plasmid conjugation simulation}
%------------------------------------------------------------------------------


\begin{document}
%-------------------------------------------------------------
% Title
%-------------------------------------------------------------
\title{
  \textcap{Laboratory Notebook \\[1cm]
  \textaut{Beginning 05-29-2016}}
}

\author{
  \textaut{JD Russo}\\ \\
  Bucknell University
}

\date{}

\maketitle

\printindex
\tableofcontents
% \newpage

\pagestyle{scrheadings}
%-------------------------------------------------------------


%-------------------------------------------------------------
% Begin journal entries
%-------------------------------------------------------------
\labday{June 1, 2016}
\obj{Refine Gillespie algorithm proof-of-concept, and continue background research.}

\experiment{gillespie}
After reviewing my first version of my code\footnote{Git commit 4105445}, JJ had
feedback on some changes to make, which I'll go through and fix one by one.

\begin{itemize}
  \item[$\Rightarrow$] $\tau$ should be generated with a different random number than the number used
  for determining the reaction.
  \item Changed $\tau$ to use a new, freshly generated number.

  \item[$\Rightarrow$] An unnecessary division in the code when calculating the
  propensity array can be removed for efficiency.
  \item Removed the unnecessary division and timed execution with and without
  using the \texttt{time} command. Too much variation in execution times to determine
  any significant change.

  \item[$\Rightarrow$] How would I modify the array of propensity functions if,
  for example, reaction 2 were changed from $X \rightarrow 2X$ to $2X \rightarrow 3X$?
  Reactions like this can be found in sexual reproduction and chemical reactions.
  \item I need to look into this more. My first guess would be maybe I need to
  square the propensity or probability of that reaction, since two reagents are involved.
\end{itemize}

Before I spend more time trying to answer the last point, I'm going to review some
more background materials. I'll start by reading the Wikipedia page on antimicrobial
resistance, and go from there.

%-------------------------------------------------------------
\labday{June 2, 2016}

\obj{Continue background research on antimicrobial resistance and plasmids.}

\section{Background}
\subsection*{Antimicrobial Resistance}

\begin {itemize}[label={--}]
\item Broader topic than just bacteria and antibiotics
\item 3 primary ways resistance arises in bacteria
\begin{itemize}
  \item Natural resistances
  \item Genetic mutation
  \item Acquiring resistance from another species \textbf{What we're studying with plasmids}
\end{itemize}
\item All classes of microbes develop resistances
\item Present in all parts of the world
\end{itemize}


\subsection*{Mutation}
\begin{itemize}
  \item Low probability of mutating resistances
  \item Some mutations can produce enzymes that render antibiotics inert
  \item Some mutations can eliminate the targets of certain antibiotics
\end{itemize}


\subsection*{Acquired Resistances}
\begin{itemize}
  \item \textit{Conjugation} is passing genetic material from one bacteria to another via
  \textbf{plasmids} and \textbf{transposons}. This requires physical contact
  \item \textit{Transformation} is when a bacteria absorbs genetic material from
  a free plasmid in its environment
  \item Viruses can also take genetic material from a bacterium and inject it into other bacteria
  \item \textbf{Horizontal} transfer is sharing genetic material with other bacteria
  \item \textbf{Vertical} transfer is sharing genetic material through reproduction with daughter cells
  \item \textbf{Transposons} are small amounts of DNA that can move between genetic
  elements like chromosomes and plasmids.
\end{itemize}


\subsection*{Plasmids}
\begin{itemize}
  \item Cannot reproduce on their own, without a host bacterium
  \item Widely distributed
  \item Bacteria can hold many plasmids per cell
  \item Plasmid genomes have \textit{core} genes for transmission and replication
  and \textit{accessory} genes that encode traits.
\end{itemize}


\subsection*{The Plasmid Paradox}
"Plasmids impose a fitness cost on their bacterial hosts that generates selection
against plasmid carriage under conditions in which plasmid genes do not provide
any benefit to the host ... The great irony of the plasmid paradox is that exposure
to conditions that select for plasmid-carried genes can also ultimately lead to
plasmid loss."
\begin{itemize}
  \item Recurrent horizontal transfer
  \item Genes encoded
\end{itemize}

\section*{Skype with Prof. Dong}
\begin{itemize}
  \item For the $2X \rightarrow 3X$ reaction mentioned in yesterday's entry, the
  propensity array element for the new reaction must be multiplied by two. Since
  it involves two reagents now, it's twice as probable. \textit{(Why doesn't this
  make it half as probable?)}

  \item The theoretical line should follow the solution to the ODE, which in the
  simplest case is just the exponential.

  This should be plotted on a semi-log
  plot (meaning logarithmic y-axis) since it's an exponential function. The slope
  of this line will be the exponent.

  \item It would be helpful to average the five independent Gillespie algorithm
  runs that are executed, and compare the average line to the theoretical line.

  However, the Gillespie algorithm has an element of randomness in the time steps
  it takes. Therefore, between separate runs, times of datapoints will not line up.
  Without having points with the same time, we can't meaningfully average them.
  We're not sure what the answer is to this yet, but we're putting it on hold for now.

  We also discussed the central limit theorem, and how if you increase the population
  size in the simulation, the simulation runs become far closer to the theoretical
  model. As the population grows, the standard deviation of the mean is reduced,
  and the Gaussian that describes the distribution of the runs gets narrower.

  In biology, the opposite is often the case, where very small populations lead
  to wide Gaussians, and much noisier data. For example, in our research, if a parent
  cell has 5 plasmids and produces two daughters, it's likely that the plasmids
  will be distributed 3 to one, 2 to the other. However, that's just likely, not
  guaranteed. If this doesn't happen, and if for example one daughter receives all
  5 and the other none, this small initial difference can hugely affect the system
  as it grows.

  \item We would like to start discussing a journal article every week to become
  more familiar with the literature surrounding this topic. To that end, I've
  been provided with a first article to read\cite{thattai}.
\end{itemize}

%------------------------------------------------------------------------------
%	BIBLIOGRAPHY
%------------------------------------------------------------------------------
\begin{thebibliography}{2}
% \bibitem{lamport94}
% Leslie Lamport,
% \emph{\LaTeX: A Document Preparation System}.
% Addison Wesley, Massachusetts,
% 2nd Edition,
% 1994.
\bibitem{thattai}
Mukund Thattai, Alexander van Oudenaarden
\emph{Intrinsic noise in gene regulatory networks}
PNAS vol. 98 no. 15
2001.
http://www.pnas.orgycgiydoiy10.1073ypnas.151588598/
\end{thebibliography}
%------------------------------------------------------------------------------

\end{document}
